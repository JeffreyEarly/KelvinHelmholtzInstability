\documentclass[11pt]{amsart}
\usepackage{geometry}                % See geometry.pdf to learn the layout options. There are lots.
\geometry{letterpaper}                   % ... or a4paper or a5paper or ... 
%\geometry{landscape}                % Activate for for rotated page geometry
\usepackage[parfill]{parskip}    % Activate to begin paragraphs with an empty line rather than an indent
\usepackage{graphicx}
\usepackage{amssymb}
\usepackage{epstopdf}
\DeclareGraphicsRule{.tif}{png}{.png}{`convert #1 `dirname #1`/`basename #1 .tif`.png}

\title{Modeling the Kelvin-Helmholtz Instability}
\author{Jeffrey J. Early}
%\date{}                                           % Activate to display a given date or no date

\begin{document}
\maketitle

\section{Governing Equations}

An incompressible two-dimensional fluid is governed by,
\begin{align}
\label{x-mom}
\frac{\partial u}{\partial t} + u \frac{\partial u}{\partial x} + v \frac{\partial u}{\partial y} &= -\frac{1}{\rho_0} \frac{\partial p}{\partial x} + \nu \nabla^2 u\\
\label{y-mom}
\frac{\partial v}{\partial t} + u \frac{\partial v}{\partial x} + v \frac{\partial v}{\partial y} &= -\frac{1}{\rho_0} \frac{\partial p}{\partial y} + \nu \nabla^2 v\\
\label{cont}
\frac{\partial u}{\partial x} + \frac{\partial v}{\partial y} &= 0.
\end{align}

If you take the vertical component of the curl of the above equations, i.e., $\frac{\partial}{\partial x} (2) - \frac{\partial}{\partial y} (1)$, you find that
\begin{equation}
\label{vorticity}
\left( \frac{\partial}{\partial t} + u\frac{\partial}{\partial x} + v\frac{\partial}{\partial x} - \nu \nabla^2 \right)\left( \frac{\partial v}{\partial x} - \frac{\partial u}{\partial y} \right) = 0
\end{equation}

If define a stream function $\psi$ such that
\begin{equation}
u = - \frac{\partial \psi}{\partial y}, v = \frac{\partial \psi}{\partial x}
\end{equation}
then the incompressibility condition \ref{cont} is automatically satisfied and the governing equation is a function of only $\psi$ and $p$. In addition, however, we can write equation \ref{vorticity} as
\begin{equation}
\nabla^2 \psi_t + J \left( \psi, \nabla^2 \psi \right) - \nu \nabla^4 \psi = 0
\end{equation}
where $J(a, b) = a_x b_y - a_y b_x$.

More explicitly this is,
\begin{equation}
\nabla^2 \psi_t + \psi_x \nabla^2 \psi_y - \psi_y \nabla^2 \psi_x - \nu \nabla^4 \psi = 0
\end{equation}

\section{Initial Conditions}

We want to set the initial conditions to $v=0$ and,
\begin{equation}
u_0(y) = \begin{cases}
U_0      &  y>a, \\
\frac{U_0}{a}y      & -a < y < a, \\
- U_0  & y<-a
\end{cases}
\end{equation}
in a domain periodic in $x$ but bounded by walls at $y=b$ and $y=-b$. Integrating this initial condition to provide an initial condition for the streamfunction $\psi$ we find that,
\begin{equation}
\psi_0(y) = \begin{cases}
U_0 \left( b - y \right)      &  y>a, \\
U_0 \left( - \frac{y^2}{2a} + b - \frac{a}{2} \right)    & -a < y < a, \\
U_0 \left( b + y \right)  & y<-a.
\end{cases}
\end{equation}

According to linear stability analysis, the fastest growing wavenumber $k$ for this system occurs when $ka=0.398$. This corresponds to a wavelength of $\lambda = \frac{2 \pi}{k} = 2 \pi a \cdot 2.51$. The domain width should therefore be several factors larger than this wavelength.

Reasonable setting would therefore be a domain height of $2b$, width of $8b$, $a=0.1b$ and $U_0=1$ m/s. Try $b=10$ meters.

\section{Transforms}

\begin{align}
\psi \mapsto& \left( e^{i k x}, \sin (l y) \right) \\
\zeta \mapsto& \left( e^{i k x}, \sin (l y) \right) \\
u \mapsto& \left( e^{i k x}, \cos (l y) \right) \\
v \mapsto& \left( e^{i k x}, \sin (l y) \right) \\
\end{align}

u needs to be in a cosine basis in the y, psi needs a sine basis. 

\end{document}  